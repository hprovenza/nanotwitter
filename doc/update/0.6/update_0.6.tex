\documentclass[]{article}
\usepackage{fancyhdr}
\usepackage{caption}

\pagestyle{fancy}
\fancyhf{}
\lhead{COSI105b}
\chead{nanoTwitter v0.6 update}
\rhead{Team Reptile Splash}

\begin{document}
	We have successfully implemented redis on our Heroku server side. 
        So far, we are only been using redis to cache the most recent tweet on 
        the index page.\\
        The test result for the index page before and after caching is shown below.
        \captionof{table}{Loader test using 360 clients on index page for 1 min}
        \begin{center}
        \begin{tabular}{|c|c|c|c|}
	\hline
	  & Average response time & Success & Timeout\\ \hline
	Before Caching & 8006ms & 1260 & 1320\\
	After Caching & 7512ms & 3943 & 0 \\ \hline
	\end{tabular} \\
        \end{center}
        Caching has shown to have improved the performance of the index page.\\
        It took us some time to figure out how to use redis in our application. And we still haven't 
        successfully made redis to work locally yet.\\
        In terms of future plans, 
        we are planning to use caching for other parts of our application. 
        For example, we can cache the timeline for each user. 
        In addition, we will conduct more tests on such routes, and compare the results of using and 
        not using caching.\\
\end{document}
